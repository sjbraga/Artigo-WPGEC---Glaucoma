\documentclass[conference]{IEEEtran}

% *** GRAPHICS RELATED PACKAGES ***
%
\ifCLASSINFOpdf
   \usepackage[pdftex]{graphicx}
  % declare the path(s) where your graphic files are
  % \graphicspath{{../pdf/}{../jpeg/}}
  % and their extensions so you won't have to specify these with
  % every instance of \includegraphics
  % \DeclareGraphicsExtensions{.pdf,.jpeg,.png}
\else
  % or other class option (dvipsone, dvipdf, if not using dvips). graphicx
  % will default to the driver specified in the system graphics.cfg if no
  % driver is specified.
  % \usepackage[dvips]{graphicx}
  % declare the path(s) where your graphic files are
  % \graphicspath{{../eps/}}
  % and their extensions so you won't have to specify these with
  % every instance of \includegraphics
  % \DeclareGraphicsExtensions{.eps}
\fi

% *** MATH PACKAGES ***
%
\usepackage{amsmath}


% *** PDF, URL AND HYPERLINK PACKAGES ***
%
%\usepackage{url}


\usepackage[brazilian]{babel}
\usepackage[utf8]{inputenc}
%\usepackage[T1]{fontenc}
\usepackage{fancyhdr}


% correct bad hyphenation here
%\hyphenation{op-tical net-works semi-conduc-tor}


\pagestyle{fancy}
%\fancyhf{}
\chead{VII Workshop de P\'{o}s-Gradua\c{c}\~{a}o - Engenharia de Computa\c{c}\~{a}o - WPGEC 2018}
\renewcommand{\headrulewidth}{2pt}

\pagenumbering{gobble}

\begin{document}

%
% paper title
% Titles are generally capitalized except for words such as a, an, and, as,
% at, but, by, for, in, nor, of, on, or, the, to and up, which are usually
% not capitalized unless they are the first or last word of the title.
% Linebreaks \\ can be used within to get better formatting as desired.
% Do not put math or special symbols in the title.
\title{Paper title - English \\ T\'{i}tulo do artigo - somente para documento escrito em Portugu\^{e}s}



% conference papers do not typically use \thanks and this command
% is locked out in conference mode. If really needed, such as for
% the acknowledgment of grants, issue a \IEEEoverridecommandlockouts
% after \documentclass

% for over three affiliations, or if they all won't fit within the width
% of the page, use this alternative format:
% 
\author{\IEEEauthorblockN{BRAGA, S. J.\IEEEauthorrefmark{1};
GOMI, E. S.\IEEEauthorrefmark{1}}
\IEEEauthorblockA{\IEEEauthorrefmark{1}Escola Politécnica da Universidade de São Paulo}}


% make the title area
\maketitle

\thispagestyle{fancy}

% As a general rule, do not put math, special symbols or citations
% in the abstract
\renewcommand{\abstractname}{Abstract}
\begin{abstract}
Abstract here.
\end{abstract}

\renewcommand\IEEEkeywordsname{Keywords}
\begin{IEEEkeywords}
\label{Keywords}
word 1; word 2.
\end{IEEEkeywords}

%%%%Only for english article%%%%%%%%%%%%%%%%
\renewcommand\IEEEkeywordsname{Classification}
\begin{IEEEkeywords}
	\label{Classification}
	indicate whether the study is: undergraduate research, master's degree or doctorate degree [Only for article in English].
\end{IEEEkeywords}
\renewcommand\IEEEkeywordsname{Category}
\begin{IEEEkeywords}
	\label{Category}
	Indicate the state of the research (applies to Master degree/Doctorate degree): 
	Beginner, Intermediate ou In conclusion [Only for article in English] 
\end{IEEEkeywords}
%%%%%%%%%%%%%%%%%%%%%%%%%%%%%%%%%%%%%%%%%
\renewcommand{\abstractname}{Resumo}
\begin{abstract}
\label{Resumo}
\'E necess\'aria a inser\c{c}\~{a}o do resumo para artigo escrito em Portugu\^{e}s.
\end{abstract}

\renewcommand\IEEEkeywordsname{Palavras-chave}
\begin{IEEEkeywords}
\label{Palavras-chave}
palavra 1; palavra 2.
\end{IEEEkeywords}

\renewcommand\IEEEkeywordsname{Classifica\c{c}\~{a}o}
\begin{IEEEkeywords}
	\label{classificacao}
	Indicar se o estudo \'{e}: Incia\c{c}\~{a}o cientifica, Mestrado ou Doutorado
\end{IEEEkeywords}

\renewcommand\IEEEkeywordsname{Categoria}
\begin{IEEEkeywords}
	\label{Categoria}
	Indicar o estado da pesquisa (aplica para Mestrado / Doutorado): 
 	Iniciante, Intermedi\'{a}ria ou Em conclus\~{a}o 
\end{IEEEkeywords}
% For peer review papers, you can put extra information on the cover
% page as needed:
% \ifCLASSOPTIONpeerreview
% \begin{center} \bfseries EDICS Category: 3-BBND \end{center}
% \fi
%
% For peerreview papers, this IEEEtran command inserts a page break and
% creates the second title. It will be ignored for other modes.
\IEEEpeerreviewmaketitle


\section{Introdução}

\section{Diagnóstico de glaucoma}

\section{Redes neurais profundas}

\section{Experimentos e resultados}

  \subsection{Dataset}

  O dataset original foi obtido com o departamento de oftalmologia da Unicamp. O dataset consiste de imagens de OCT de 56 olhos com glaucoma e 66 olhos normais, totalizando 122 pacientes. Os gráficos de espessura de fibras nervosas foram obtidos através da extração das imagens do PDF do exame. Foram selecionados para o experimento somente os olhos de pacientes que foram manualmente classificados por especialistas.

  Para a separação do dataset em treino e validação, foram separados 20\% de olhos normais e 20\% de olhos com glaucoma para validação, e o restante para treino, totalizando 98 imagens de treino e 24 para validação. As imagens selecionadas para teste não estão presentes no dataset de treino, para que o algoritmo possa classificar imagens ainda não vistas.

  Para evitar overfitting, foi empregada uma técnica para aumentar o número de exemplos a partir das imagens no dataset de treino. Cada imagem foi rotacionada 100 vezes em ângulos aleatórios entre 0 e 360 graus, gerando assim um dataset de treino com 9800 imagens. As imagens de validação não foram rotacionadas.

%aumento do dataset, rotações aleatorias entre 0 e 360
%divisao treino e validação

  \subsection{Pré-processamento}

  Para utilização do transfer learning, foi necessário fazer a subtração do pixel médio em todas as imagens do dataset de treino. O valor médio de cada pixel da imagem é calculado sobre todas as imagens do dataset de treino. Essa imagem média é então subtraída de cada imagem do dataset. Dessa forma, todos os pixels de entrada estarão no mesmo range(?), evitando divergência nos gradientes.

  Onde houveram falhas na aquisição da imagem, pixels com valores RGB próximos ao preto foram substituídos pelo valor de preto absoluto RGB (0, 0, 0).

  % subtração da media
  % areas pretas para zero absoluto

  \subsection{Experimentos}
%setup do servidor, quais gpus
  %utilizando caffe

  \subsubsection{Resultados com transfer learning}

%estrategia de learning rate
  %iterações e tempo de processamento
  %grafico da acuracia

  \subsubsection{Resultados sem transfer learning}

%iterações e tempo de processamento
  %grafico da acuracia

\section{Discussão}

%dificuldades para definir os parametros corretos de treinamento
%dataset pequeno
%tempo de treinamento(?)

\section{Conclusão}



%Use BIB file
% \bibliographystyle{abntex2-num}
% \bibliography{template}

% that's all folks
\end{document}


